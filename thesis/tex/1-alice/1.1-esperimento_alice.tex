%%%%%%%%%%%%%%%%%%%%%%%%%%%%%%%%%
\section{L'esperimento ALICE}
L'esperimento \emph{A Large Ion Collider Experiment} (ALICE) \cite{The_ALICE_Collaboration_2008} fa parte del complesso del \emph{Large Hadron Collider} (LHC), il più grande e più potente acceleratore di particelle al mondo \cite{Lyndon_Evans_2008}.
L'LHC è situato sul confine franco-svizzero nella regione di Ginevra, e si trova all'interno di un tunnel sotterraneo che raggiunge una profondità fino a 175 m.
Lungo l'anello dell'LHC sono stati realizzati quattro principali esperimenti: ALICE, CMS, LHCb e ATLAS.
Essi sono disposti secondo lo schema riportato in \autoref{fig:LHC}.
\begin{figure}[htb]
    \centering
    \includesvg[width=0.7\textwidth]{image/1-alice/LHC.svg}
    \caption{Rappresentazione schematica delle posizioni dei quattro grandi esperimenti di LHC, indicati dai punti gialli. "p" e "Pb" indicano gli acceleratori lineari dei protoni e degli ioni di Pb, PS è il \emph{Proton Synchrotron}, SPS il \emph{Super Proton Synchrotron}. (Autore: \href{https://it.m.wikipedia.org/wiki/File:LHC.svg}{Arpad Horvath}).}
    \label{fig:LHC}
\end{figure}
Il centro di ricerca responsabile dell'LHC è l'\emph{Organizzazione Europea per la Ricerca Nucleare} (CERN), fondato nel 1954 a Ginevra, in Svizzera, e fin da allora si è occupato della ricerca nel campo della fisica nucleare e subnucleare.
L'LHC non è solamente il più grande apparato sperimentale al mondo per dimensione, ma anche per numero di fisici e altri studiosi coinvolti, provenienti da qualunque angolo del mondo, rendendo il CERN senza dubbio una delle collaborazioni scientifiche più importanti del mondo.   
Tuttavia è da menzionare che l'LHC non è sempre in operatività, ma lo è solamente in alcuni periodi chiamati \emph{Run}, i quali durano anni, alternati a periodi di manutenzione e aggiornamento, chiamati \emph{Long Shutdowns} (LS).
Di \textit{Run} ad oggi ce ne sono stati tre (Run 1 2009-2013, Run 2 2015-2018, Run 3 2022-2025) e futuri \textit{Run} sono già stati programmati.
In questa tesi si considereranno i dati raccolti durante il \textit{Run} 2 di LHC.
L'LHC è capace di accelerare protoni e anche ioni pesanti (per esempio Pb) a energie ultra-relativistiche.
Questa versatilità permette lo studio delle particelle e delle loro interazioni ad energie estreme.

L'esperimento ALICE si occupa proprio dello studio della materia adronica nella fase detta Plasma di Quark e Gluoni (QGP), uno stato della materia nel quale i quark e i gluoni sono in uno stato non confinato, ottenuto con valori estremi di pressione e temperatura.
Il QGP è ritenuto essere stato la fase in cui si trovava la materia negli istanti iniziali dell'universo, prima che si formasse la materia adronica ordinaria.
Il QGP risulta essere un sistema composto da particelle che interagiscono per interazione forte, descritta dalla teoria della \emph{Quantum Chromodynamics} (QCD) del modello standard.
Perciò studiare le proprietà di questa fase rende l'esperimento ALICE insieme agli altri grandi esperimenti ad LHC, fondamentale per lo studio dei processi QCD.

In LHC, le particelle cariche vengono accelerate tramite cavità di radio-frequenza (cavità RF) che sono situate lungo la circonferenza dell'LHC e devono operare con una frequenza ben precisa in modo che le particelle possano essere accelerate correttamente.
Le particelle vengono prima accelerate da ferme al di fuori dell'LHC, formando un fascio continuo di particelle.
Successivamente, arrivati a una particolare energia (per un fascio di protoni $\sim 50$ MeV) le particelle vengono separate in pacchetti, e dopo ulteriori accelerazioni vengono immesse nell'LHC.
%%%%