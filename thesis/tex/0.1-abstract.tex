La produzione di (anti)deuteroni è di grande interesse nel campo della fisica delle alte energie, perché permette di ottenere informazioni che spaziano da processi QCD come l'adronizzazione dei quark, ai modelli cosmologici sull'origine dell'universo e allo studio della materia oscura.

In questa tesi si è utilizzato PYTHIA come simulatore Monte Carlo (versione 8.312), attraverso il quale si è studiata la formazione di (anti)deuteroni in collisioni pp ad energie del centro di massa $\sqrt s = 13$ TeV.
In particolare è stato effettuato uno studio sui vari canali di produzione del (anti)deuterone e sull'ottimizzazione di alcuni parametri interni di PYTHIA, come il fattore di normalizzazione, mettendo in confronto i dati raccolti dall'esperimento ALICE con quelli generati da PYTHIA.

I risultati hanno mostrato una riproduzione qualitativamente fedele dei dati di ALICE; tuttavia sono ancora presenti dei disaccordi che suggeriscono la necessità di ulteriori miglioramenti del modello. 